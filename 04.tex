\chapter{September 29 - October 4}

\section{Orthogonal Matrices}

Let $T : \R^n \to \R^n$ be an orthogonal linear transformation:

\begin{definition}
  We say $B$ is an \emph{eigenbasis} for $T$ if $B$ is an orthonormal basis of eigenvectors of $T$.
\end{definition}

\begin{remark}
  If $n=1$, $T$ is one of the following:
    \begin{align*}\bmatr{1} & & \bmatr{-1}\end{align*}
\end{remark}

\begin{remark}
  If $n=2$, and $A$ is the matrix for $T$, $A$ must be a real matrix satisfying:
    $$A\transpose{A} = \transpose{A}A = I$$
  and since $\setof{(1,0), (0,1)}$ is an orthonormal basis we must have:
    $$\norm{A\pmatr{1 \\ 0}} = \norm{A\pmatr{0 \\ 1}} = 1$$
  thus $A$ must be of the following form for some $\theta \in [0,2\pi)$:
  $$A = \pmatr{
    \cos \theta & \pm \sin \theta \\
    \sin \theta & \mp \cos \theta
  }$$
\end{remark}

\begin{remark}
  If $n=2$, by lifting $T$ to being a unitary transformation $\C^n \to \C^n$,
  we can distinguish between rotations and reflections from the eigenvectors and eigenvalues of $A$.
  We know the eigenvalues must be complex numbers of length 1, so if they are real, they are $\pm 1$.
  So let $A$ be the matrix of $T$ under an eigenbasis, it must be of the form:
    \begin{center}
      \begin{tabular}{c c c}
        $\bmatr{
          \pm 1 & 0 \\
          0 & \pm 1
        }$ & or & $\bmatr{
          \pm 1 & 0 \\
          0 & \mp 1
        }$ \\
        rotation & & reflection
      \end{tabular}
    \end{center}
  Otherwise, if the eigenvalues are not real, they are of the form $\cos \theta + i \sin \theta$ for some $\theta \in [0,2\pi)$
  Matrices of the form:
    $$A = \pmatr{\cos \theta & -\sin \theta \\ \sin \theta & \cos \theta}$$
  have eigenvalues $\cos \theta \pm i \sin \theta$ and are rotations.
\end{remark}
\begin{remark}
  An orthogonal 2x2 matrix can be the composition of a rotation and a reflection.
\end{remark}

\begin{theorem}
  Let $A$ be a real, orthogonal, $n \times n$ matrix. Then $A$ is block diagonal with blocks of size $0$ or $1$.
\end{theorem}
\begin{proof}
  Lift $A$ to a $n \times n$ complex, unitary matrix. Then, since the entries are real
    $$A\vec x = \lambda \vec x \text{ for } \vec x \ne 0 \implies A\vec x = \conj{\lambda}\conj{\vec x}$$
  Thus non-real eigenvalues come in conjugate pairs. Since $A$ is unitary as a complex matrix,
  we can find an \emph{eigenbasis} $B$ of $\C^n$ for $A$. Then consider an arbitrary pair of eigenvalues $\vec v$ and $\vec w = \conj{\vec v}$.
  We want to find two real vectors $\vec x, \vec y \in \R^n$ such that $\spanof{\vec x, \vec y} = \spanof{\vec v, \vec w}$ over $\C$.
  So define
    \begin{align}
      \vec x &=  \vec v +  \vec w  & &(= 2\Re(\vec v)) \\
      \vec y &= i\vec v + i\vec w  & &(= -2\Im(\vec v))
    \end{align}
  Clearly, by definition, $\vec x, \vec y \in \spanof{\vec v,\vec w}$, and furthermore we have:
    \begin{align}
      \vec v &= \frac{1}{2i}\paren{i\vec x + \vec y} \\
      \vec w &= \frac{1}{2i}\paren{i\vec x - \vec y} \\
    \end{align}
  Thus $\vec w, \vec w \in \spanof{\vec v,\vec w}$. Applying Gram-Schmidt allows us to turn $\setof{\vec x, \vec y}$
  into a real orthonormal basis of $\spanof{\vec v, \vec w}$. Doing this for every conjugate pair of non-real $\vec v_i \in B$
  gives us a new, real orthonormal basis $B'$ such that:
  $$[A]_{b'} = \pmatr{
    (2\times 2) & 0 & \ldots & 0 \\
    0 & (2 \times 2) & \ldots & 0 \\
    \vdots & \vdots & \ddots & \vdots \\
    0 & 0 & \ldots & (1\times 1)
  }$$
  where each block is also orthogonal matrix.
\end{proof}
\begin{remark}
  This means that any orthogonal transformation $T$, when viewed under the right basis,
  is a collection of pairwise orthogonal rotations ($2 \times 2$ blocks) together with some fixed and reflected lines ($\pm1$ eigenvalues).
\end{remark}
\begin{thmexample}
  In $\R^3$, an orthogonal matrix $A$ may look like:
  $$A = \bmatr{
    \cos \frac{\pi}3 & -\sin \frac{\pi}3 & 0 \\
    \sin \frac{\pi}3 &  \cos \frac{\pi}3 & 0 \\
    0                &  0                & 1
  }$$
\end{thmexample}

\section{Rigid Motions}
\begin{definition}
  A \emph{rigid motion} is a function $f : \R^n \to \R^n$ such that
    $$\norm{\vec x - \vec y} = \norm{f(\vec x) - f(\vec y)} \text{ for all } \vec x, \vec y \in \R^n$$
  That is, $f$ preserves distances.
\end{definition}
\begin{defexample}
  A translation $\setof{ \vec x \mapsto \vec x + \vec a }$ is a \emph{rigid motion}
\end{defexample}
\begin{defexample}
  An orthogonal linear transformation is a \emph{rigid motion}
\end{defexample}

\begin{theorem}
  Any \emph{rigid motion} $f : \R^n \to \R^n$ can be written uniquely as
    $$f = g \circ T$$
  where $g$ is a translation and $T$ is an orthogonal linear transformation
\end{theorem}
\begin{proof}
  Define $T : \R^n \to \R^n$ by
    $$T(\vec x) = f(\vec x) - f(\vec 0)$$
  $T$ is clearly a \emph{rigid motion}, and $T(\vec 0) = f(\vec 0) - f(\vec 0) = \vec 0$.
  Also $f = g \circ T$ where $g$ is the translation $g(\vec x) = \vec x + f(\vec 0)$.
  We will prove that $T$ is linear.
  First observe that, for any $\vec x \in \R^n$
    $$\norm{T \vec x} = \norm{T \vec x - T \vec 0} = \norm{\vec x - \vec 0} = \norm{\vec x}$$
  Next we will show that, for any $\vec x, \vec y \in \R^n$
    $$\inp{T\vec x}{T\vec y} = \inp{\vec x}{\vec y}$$
  this is true because
    \begin{align}
      \norm{T\vec x - T\vec y}^2 
        &= \inp{T\vec x - T\vec y}{T \vec x - T \vec y} \\
        &= \inp{T \vec x}{T \vec x} - \inp{T \vec x}{T \vec y} - \inp{T \vec y}{T \vec x} + \inp{T\vec y}{T \vec y} \\
        &= \norm{T\vec x}^2 + \norm{T \vec y}^2 - 2\inp{T \vec x}{T \vec y}
        &= \norm{\vec x}^2 + \norm{\vec y}^2 - 2\inp{T \vec x}{T \vec y}
    \end{align}
  but also
    \begin{align}
      \norm{T \vec x - T\vec y}^2
        &= \norm{\vec x - \vec y}^2 \\
        &= \norm{\vec x}^2 + \norm{\vec y}^2 - 2\inp{\vec x}{T \vec y}
    \end{align}
  subtracting these two equations from each other we obtain
    $$\inp{T \vec x}{T \vec y} = \inp{\vec x}{\vec y}$$
  Using this fact we can show the two properties of linearity:
  \begin{enumerate}[i.]
    \item For any $a \in \R$, $\vec x \in \R^n$:
      \begin{align}
        \norm{T(a \vec x) - aT(\vec x)}^2
          &= \norm{T(a\vec x)}^2 + \norm{aT(\vec x)}^2 - 2\inp{T(a \vec x)}{aT(\vec x)} \\
          &= \norm{a\vec x}^2 + a^2\norm{\vec x}^2 - 2a\inp{a \vec x}{\vec x} \\
          &= 2a^2\norm{\vec x}^2 - 2a^2\norm{\vec x}^2 = 0
      \end{align}
      Thus, by positive definiteness of the norm $T(a\vec x) = aT(\vec x)$.

    \item For any $\vec x, \vec y \in \R^n$:
      \begin{align}
        \norm{T(\vec x + \vec y) - T(\vec x) - T(\vec y)}^2
          &= \norm{T(\vec x + \vec y)}^2 + \norm{T(\vec x)}^2 + \norm{T(\vec y)}^2 \\
          &\phantom{=} - 2\inp{T(\vec x + \vec y)}{T(\vec x)} - 2\inp{T(\vec x + \vec y)}{T(\vec y)} \notag \\
          &\phantom{=} + 2\inp{T(\vec x)}{T(\vec y)} \notag \\
          &= \norm{\vec x + \vec y}^2 + \norm{\vec x}^2 + \norm{\vec y}^2 \\
          &\phantom{=} -2\inp{\vec x + \vec y}{\vec x} - 2\inp{\vec x + \vec y}{\vec y} + 2\inp{\vec x}{\vec y} \notag \\
          &= \norm{\vec x}^2 + \norm{\vec y}^2 + 2\inp{\vec x}{\vec y} -\norm{\vec x + \vec y}^2 = 0
      \end{align}
      Thus, by positive definiteness of the norm $T(\vec x + \vec y) = T(\vec x) + T(\vec y)$.
  \end{enumerate}
  Thus $T$ is a linear \emph{rigid motion}, so $T$ is orthogonal. It remains only to be shown that $f = g \circ T$ is unique.
  Suppose $f = g' \circ T'$ for a translation $g'$ and orthogonal transformation $T'$, then:
    \begin{align}
      f(\vec 0) &= (g' \circ T') \vec 0 = g'(\vec 0) \\
                &= (g \circ T) \vec 0 = g(\vec 0)
    \end{align}
  Thus $g(\vec x) = g'(\vec x) = \vec x + f(\vec 0)$. But then
    $$T'\vec x = \paren{g^{-1} \circ f}\vec x = f(\vec x) - f(\vec 0) = T\vec x$$
  so $g' = g$ and $T' = T$ as required for uniqueness.
\end{proof}
